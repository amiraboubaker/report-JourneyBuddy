\section{Introduction}

Ce chapitre présente les différentes itérations de développement du projet sous forme de \textbf{quatre sprints essentiels}, chacun correspondant à un ou plusieurs cas d'utilisation majeurs. Chaque sprint comprend les étapes de planification, conception, modélisation UML, et développement avec les interfaces graphiques correspondantes.

\section{Sprint 1 : Inscription, Authentification et Gestion de profil}

Ce premier sprint a pour objectif la mise en place des fonctionnalités fondamentales d'accès à l'application : inscription, authentification et gestion du profil utilisateur

\begin{table}[H]
\centering
\begin{tabularx}{\textwidth}{|c|l|l|X|c|}
\hline
\textbf{Sprint} & \textbf{User Story} & \textbf{En tant que} & \textbf{Je veux} & \textbf{Priorité} \\
\hline
Sprint 1 & Inscription & Voyageur & Créer un compte pour personnaliser mon expérience & Haute \\
\hline
Sprint 1 & Authentification & Voyageur & M’authentifier afin d’accéder à mon espace personnel. & Haute \\
\hline
Sprint 1 & Gestion de profil & Voyageur & Modifier mes informations de profil (photo, téléphone, ville/pays, bio, adresse, mot de passe) & Moyenne \\
\hline
\end{tabularx}
\caption{Sprint Backlog du Sprint 1}
\end{table}

% ================================
% 1) Use Case
% ================================
\subsection{Diagramme de cas d’utilisation du Sprint 1}

Ce diagramme illustre les interactions principales entre les acteurs et les fonctionnalités du sprint 1 (inscription, authentification et gestion du profil).

\begin{figure}[H]
    \centering
    \includegraphics[width=0.8\textwidth]{diagrams/usecases/profile.png}
    \caption{Diagramme de cas d'utilisation - Sprint 1}
\end{figure}

% ================================
% 2) Conception détaillée
% ================================
\subsection{Conception détaillée}

\subsubsection{Diagramme de classe}

Le diagramme de classe structure les entités principales impliquées dans les opérations d’identification et de gestion du profil utilisateur.

\begin{figure}[H]
    \centering
\includegraphics[width=0.95\textwidth]{diagrams/classes/profile.png}
    \caption{Diagramme de classe - Sprint 1}
\end{figure}


\subsubsection{Diagramme de séquence}

\subsubsubsection{Diagramme de séquence d'inscription}

Ce diagramme présente les échanges entre les objets du système lors du processus d’inscription d’un utilisateur.

\begin{figure}[H]
    \centering
    \includegraphics[width=0.85\textwidth]{diagrams/sequences/insc}
    \caption{Diagramme de séquence - Inscription}
\end{figure}

\subsubsection{Diagramme de séquence d'authentification}

Ce diagramme présente les échanges entre les objets du système lors du processus d’authentification d’un utilisateur.

\begin{figure}[H]
    \centering
    \includegraphics[width=\textwidth]{diagrams/sequences/auth.png}
    \caption{Diagramme de séquence - Authentification}

\end{figure}

\subsubsection{Diagramme de séquence de Gestion du Profile}

Ce diagramme présente les échanges entre les objets du système lors du processus de la gestion du profile.

\begin{figure}[H]
    \centering
    \includegraphics[width=0.85\textwidth]{diagrams/sequences/profile.png}
    \caption{Diagramme de séquence - Gestion du Profile}
\end{figure}

% ================================
% 3) Développement
% ================================
\subsection{Développement du Sprint 1}

Dans cette phase, les interfaces graphiques associées aux fonctionnalités du sprint sont développées. Elles permettent à l’utilisateur de s’inscrire, de s’authentifier et de gérer son profil de manière intuitive et sécurisée.

\begin{figure}[H]
    \centering
    \includegraphics[width=0.5\textwidth, keepaspectratio]{ui/begin/lancement.jpg}
    \caption{Interface graphique - Lancement de l'application}
\end{figure}

\begin{figure}[H]
    \centering
    \includegraphics[width=0.5\textwidth, keepaspectratio]{ui/begin/Explore.jpg}
    \caption{Interface graphique - Exploration de l'application}
\end{figure}

\begin{figure}[H]
    \centering
    \includegraphics[width=0.5\textwidth, keepaspectratio]{ui/begin/get_started.jpg}
    \caption{Interface graphique - Début de l'application}
\end{figure}

\begin{figure}[H]
    \centering
    \includegraphics[width=0.5\textwidth, keepaspectratio]{ui/begin/overboading1.jpg}
    \caption{Interface graphique - Le premier écran de démarrage de l'application}
\end{figure}


\begin{figure}[H]
    \centering
    \includegraphics[width=0.5\textwidth,keepaspectratio]{ui/begin/overboarding2.jpg}
    \caption{Interface graphique - Le deuxième écran de démarrage de l'application}
\end{figure}


\begin{figure}[H]
    \centering
    \includegraphics[width=0.5\textwidth, keepaspectratio]{ui/begin/overboarding3.jpg}
    \caption{Interface graphique - Le troisième écran de démarrage de l'application}
\end{figure}

\begin{figure}[H]
    \centering
    \includegraphics[width=0.5\textwidth, keepaspectratio]{ui/begin/overboarding4.jpg}
    \caption{Interface graphique - Le quatrième écran de démarrage de l'application}
\end{figure}

\begin{figure}[H]
    \centering
    \includegraphics[width=0.5\textwidth]{ui/begin/insc.jpg}
    \caption{Interface graphique - Formulaire d'inscription}
\end{figure}

\begin{figure}[H]
    \centering
     \includegraphics[width=0.5\textwidth]{ui/begin/fill_insc.jpg}
    \caption{Interface graphique - Remplissage du Formulaire d'inscription}
\end{figure}

\begin{figure}[H]
    \centering
    \includegraphics[width=0.5\textwidth]{ui/begin/login.jpg}
    \caption{Interface graphique - Formulaire d'authentification}
\end{figure}

\begin{figure}[H]
    \centering
     \includegraphics[width=0.5\textwidth]{ui/begin/fill_login.jpg}
    \caption{Interface graphique - Remplissage du Formulaire d'authentification}
\end{figure}
\newpage
\begin{figure}[H]
    \centering
    \includegraphics[width=0.5\textwidth]{ui/begin/login_home.jpg}
    \caption{Interface graphique - Interface Home}
\end{figure}

 
\begin{figure}[H]
    \centering
    \includegraphics[width=0.5\textwidth]{ui/settings/profile.jpg}
    \caption{Interface graphique -  profile}
\end{figure}

\begin{figure}[H]
    \centering
    \includegraphics[width=0.5\textwidth]{ui/settings/edit_profile.jpg}
    \caption{Interface graphique - Formulaire Modifier Profile}
\end{figure}

\begin{figure}[H]
    \centering
    \includegraphics[width=0.5\textwidth]{ui/settings/fill_profile.jpg}
    \caption{Interface graphique - Remplissage du Formulaire Profile }
\end{figure}

\begin{figure}[H]
    \centering
    \includegraphics[width=0.5\textwidth]{ui/settings/updated_profile.jpg}
    \caption{Interface graphique - Profile Modifié}
\end{figure}

% \begin{figure}[H]
%     \centering
%     \includegraphics[width=0.5\textwidth]{../ui/settings/password.jpg}
%     \caption{Interface graphique - Modifier Password}
% \end{figure}

% \begin{figure}[H]
%     \centering
%     \includegraphics[width=0.5\textwidth]{../ui/settings/update_password.jpg}
%     \caption{Interface graphique - Remplissage du Formulaire de Changer Password}
% \end{figure}

% \begin{figure}[H]
%     \centering
%     \includegraphics[width=0.5\textwidth]{../ui/settings/fill_password.jpg}
%     \caption{Interface graphique - Password Modifié}
% \end{figure}


% ================================
% Conclusion Sprint 1
% ================================
\subsection*{Conclusion du Sprint 1}

Le sprint 1 a permis de mettre en place les fonctionnalités essentielles d’accès à l’application, en intégrant à la fois la modélisation UML (cas d’utilisation, diagramme de classe et de séquence) et le développement des premières interfaces. Ces fondations assurent la continuité des prochains sprints.

% ================================
% Sprint 2
% ================================

\section{Sprint 2 : Création et Gestion de Voyage, Points d'intérêts et Alertes}

Ce deuxième sprint a pour objectif la mise en place des fonctionnalités principales liées à la gestion du voyage. 
Il permet aux utilisateurs de créer et gérer leurs voyages, de consulter des points d’intérêts, de gérer leurs favoris 
et de consulter les alertes disponibles dans l’application.

\usepackage{multirow} % add this in the preamble

\begin{table}[H]
\centering
\begin{tabular}{|c|l|l|l|c|}
\hline
\textbf{Sprint} & \textbf{User Story} & \textbf{En tant que} & \textbf{Je veux} & \textbf{Priorité} \\
\hline
\multirow{5}{*}{Sprint 2} & Création du voyage & Voyageur & Créer un voyage & Haute \\
\cline{2-5}
 & Gestion du voyage & Voyageur & Gérer mon voyage & Haute \\
\cline{2-5}
 & Consultation des points d'intérêts & Voyageur & Consulter des points d'intérêts & Moyenne \\
\cline{2-5}
 & Gestion des favoris & Voyageur & Gérer mes favoris & Moyenne \\
\cline{2-5}
 & Consultation des notifications internatiounaux culturels & Voyageur & Consulter des notifications internatiounaux culturels & Moyenne \\
\hline
\end{tabular}
\caption{Sprint Backlog du Sprint 2}
\end{table}


% ================================
% 1) Use Case
% ================================
\subsection{Diagramme de cas d’utilisation du Sprint 2}

Ce diagramme illustre les interactions principales entre les acteurs et les fonctionnalités du sprint 2 
(création et gestion de voyage, points d’intérêts, favoris et alertes).

\begin{figure}[H]
    \centering
    \includegraphics[width=0.85\textwidth]{diagrams/usecases/trip.png}
    \caption{Diagramme de cas d'utilisation - Sprint 2}
\end{figure}

% ================================
% 2) Conception détaillée
% ================================
\subsection{Conception détaillée}

\subsubsection{Diagramme de classe}

Le diagramme de classe structure les entités principales impliquées dans les opérations de gestion du voyage 
et des fonctionnalités associées (points d’intérêts, favoris, alertes).

\begin{figure}[H]
    \centering
    \includegraphics[width=0.95\textwidth]{diagrams/classes/trip.png}
    \caption{Diagramme de classe - Sprint 2}
\end{figure}

\subsubsection{Diagramme de séquence}

Ces diagrammes présentent les échanges entre les objets du système lors des processus de création et gestion d’un voyage, 
ainsi que la consultation des points d’intérêt et alertes.

\begin{figure}[H]
    \centering
    \includegraphics[width=0.85\textwidth]{trip}
    \caption{Diagramme de séquence - Création de voyage}
    
    \centering
    \includegraphics[width=0.85\textwidth]{recommandation .png}
    \caption{Diagramme de séquence - Recommandation des points d'intérêts }
    
    \centering
    \includegraphics[width=0.85\textwidth]{favoris.png}
    \caption{Diagramme de séquence - Gestion des favoris}
\end{figure}

% ================================
% 3) Développement
% ================================
\subsection{Développement du Sprint 2}

Dans cette phase, les interfaces graphiques associées aux fonctionnalités du sprint sont développées. 
Elles permettent à l’utilisateur de créer et gérer ses voyages, de consulter les points d’intérêts, de gérer ses favoris 
et de recevoir les alertes en toute simplicité.

\begin{figure}[H]
    \centering
    \includegraphics[width=0.6\textwidth]{ui_trip_creation}
    \caption{Interface graphique - Création d’un voyage}
    
    \centering  
    \includegraphics[width=0.6\textwidth]{ui_trip_management}
    \caption{Interface graphique - Gestion du voyage}
    
    \centering
    \includegraphics[width=0.6\textwidth]{ui_poi}
    \caption{Interface graphique - Consultation des points d’intérêts}
    
    \centering
    \includegraphics[width=0.6\textwidth]{ui_alerts}
    \caption{Interface graphique - Consultation les notifications internationaux culturels}
\end{figure}

% ================================
% Conclusion Sprint 2
% ================================
\subsection*{Conclusion du Sprint 2}

Le sprint 2 a permis de développer les fonctionnalités principales liées à la gestion de voyage, 
en intégrant la modélisation UML (cas d’utilisation, diagramme de classe et de séquence) et 
le développement des interfaces graphiques correspondantes. 
Il constitue une étape clé dans l’évolution de l’application en enrichissant considérablement l’expérience utilisateur.

% ================================
% Sprint 3
% ================================

\section{Sprint 3 : Recommandations AI et Chatbot}

Ce troisième sprint a pour objectif la mise en place des fonctionnalités intelligentes 
qui enrichissent l’expérience utilisateur. 
Il permet aux voyageurs de recevoir des recommandations personnalisées basées sur l’IA 
et d’interagir avec un chatbot pour obtenir de l’assistance ou des conseils de voyage en temps réel.

\begin{table}[H]
\centering
\begin{tabular}{|c|l|l|l|c|}
\hline
\textbf{Sprint} & \textbf{User Story} & \textbf{En tant que} & \textbf{Je veux} & \textbf{Priorité} \\
\hline
Sprint 3 & Consultation des recommandations AI & Voyageur & Consulter des recommandations personnalisées basées sur l’IA & Haute \\
\hline
Sprint 3 & Discussion avec un chatbot personnalisé & Voyageur & Discuter avec un chatbot intelligent pour obtenir de l’aide & Haute \\
\hline
\end{tabular}
\caption{Sprint Backlog du Sprint 3}
\end{table}

% ================================
% 1) Use Case
% ================================
\subsection{Diagramme de cas d’utilisation du Sprint 3}

Ce diagramme illustre les interactions principales entre les voyageurs et les fonctionnalités intelligentes du sprint 3 
(recommandations AI et chatbot).

\begin{figure}[H]
    \centering
    \includegraphics[width=0.85\textwidth]{chatbot}
    \caption{Diagramme de cas d'utilisation - Sprint 3}
\end{figure}

% ================================
% 2) Conception détaillée
% ================================
\subsection{Conception détaillée}

\subsubsection{Diagramme de classe}

Le diagramme de classe structure les entités principales impliquées dans la génération des recommandations 
et l’interaction avec le chatbot.

\begin{figure}[H]
    \centering
    \includegraphics[width=0.95\textwidth]{chatbot}
    \caption{Diagramme de classe - Sprint 3}
\end{figure}

\subsubsection{Diagramme de séquence}

Ces diagrammes présentent les échanges entre les objets du système lors des processus de recommandation AI 
et de dialogue avec le chatbot.

\begin{figure}[H]
    \centering
    \includegraphics[width=0.85\textwidth]{recommandation}
    \caption{Diagramme de séquence - Recommandations AI}
    
    \centering
    \includegraphics[width=0.85\textwidth]{chatbot}
    \caption{Diagramme de séquence - Interaction avec le chatbot}
\end{figure}

% ================================
% 3) Développement
% ================================
\subsection{Développement du Sprint 3}

Dans cette phase, les interfaces graphiques associées aux fonctionnalités intelligentes sont développées. 
Elles permettent à l’utilisateur de consulter facilement des recommandations personnalisées 
et d’interagir de manière fluide avec le chatbot.

\begin{figure}[H]
    \centering
    \includegraphics[width=0.85\textwidth]{ui_ai_recommendations}
    \caption{Interface graphique - Recommandations AI}
    
    \centering
    \includegraphics[width=0.85\textwidth]{ui_chatbot}
    \caption{Interface graphique - Chatbot personnalisé}
\end{figure}

% ================================
% Conclusion Sprint 3
% ================================
\subsection*{Conclusion du Sprint 3}

Le sprint 3 a introduit l’intelligence artificielle dans l’application, 
en intégrant à la fois les recommandations personnalisées et un chatbot interactif. 
Cette étape marque un tournant majeur dans l’évolution du projet, 
apportant une véritable valeur ajoutée à l’expérience utilisateur 
en rendant l’application plus intelligente et interactive.

% ================================
% Sprint 4
% ================================

\section{Sprint 4 : Finalisation UI/UX et Intégration Docker}

Ce quatrième sprint a pour objectif de peaufiner l’expérience utilisateur à travers la finalisation UI/UX et de garantir la portabilité et la mise en production fluide 
grâce à l’intégration Docker.  
Ces deux étapes viennent consolider le projet et assurer sa pérennité technique et ergonomique.

\begin{table}[H]
\centering
\begin{tabular}{|c|l|l|l|c|}
\hline
\textbf{Sprint} & \textbf{User Story} & \textbf{En tant que} & \textbf{Je veux} & \textbf{Priorité} \\
\hline
Sprint 4 & Avoir une expérience utilisateur optimisée & Voyageur & Naviguer dans l’application avec fluidité et cohérence visuelle & Haute \\
\hline
Sprint 4 & Déployer l’application dans un environnement stable & Développeur & Lancer l’application dans un conteneur portable et reproductible & Haute \\
\hline
\end{tabular}
\caption{Sprint Backlog du Sprint 4}
\end{table}

% ================================
% 1) Finalisation UI/UX
% ================================
\subsection{Finalisation UI/UX}

\textbf{Bienfaits :}
\begin{itemize}
    \item Amélioration de l’ergonomie générale et de la fluidité de navigation.
    \item Uniformisation du design (couleurs, polices, icônes) pour renforcer la cohérence visuelle.
    \item Expérience utilisateur simplifiée et adaptée à différents types d’écrans.
\end{itemize}

\textbf{Démonstration :}  
Les interfaces finales ci-dessous illustrent la cohérence graphique et l’ergonomie renforcée :

\begin{figure}[H]
    \centering
    \includegraphics[width=0.55\textwidth]{ui_final_dashboard}
    \caption{Interface finale - Tableau de bord utilisateur}
    
    \centering
    \includegraphics[width=0.55\textwidth]{ui_final_navigation}
    \caption{Interface finale - Navigation optimisée}
\end{figure}

% ================================
% 2) Intégration Docker
% ================================
\subsection{Intégration Docker}

\textbf{Bienfaits :}
\begin{itemize}
    \item Portabilité : l’application peut s’exécuter sur n’importe quel environnement supportant Docker.
    \item Reproductibilité : garantie que tous les développeurs et serveurs disposent du même environnement.
    \item Déploiement simplifié : automatisation de l’installation et de la configuration.
\end{itemize}

\textbf{Démonstration :}  
Les captures ci-dessous illustrent la mise en conteneur et le déploiement de l’application :

\begin{figure}[H]
    \centering
    \includegraphics[width=0.7\textwidth]{dockerfile}
    \caption{Exemple de Dockerfile utilisé pour la conteneurisation}
    
    \centering
    \includegraphics[width=0.7\textwidth]{docker_running}
    \caption{Application en cours d’exécution dans un conteneur Docker}
\end{figure}

% ================================
% Conclusion Sprint 4
% ================================
\subsection*{Conclusion du Sprint 4}

Le sprint 4 a permis de livrer une application complète, 
à la fois **ergonomique pour les utilisateurs finaux** et **robuste pour le déploiement technique**.  
La finalisation UI/UX a apporté une meilleure fluidité et une cohérence visuelle, 
tandis que l’intégration Docker a garanti la portabilité et la stabilité du projet.  
Ce sprint clôture le cycle de développement en assurant la qualité globale de l’application, 
prête à être utilisée et déployée dans des environnements variés.
